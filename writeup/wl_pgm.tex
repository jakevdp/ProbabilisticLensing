\documentclass[12pt]{article}
%\documentclass[preprint]{aastex}
%\documentclass[twocolumn]{emulateapj} 

\usepackage{graphicx}
\graphicspath{{../code/figures/}}

\newcommand{\eqn}[1]{eq.~(\ref{eq:#1})}
\newcommand{\eqns}[2]{eqs.~(\ref{eq:#1}) and (\ref{eq:#2})}
\newcommand{\fig}[1]{Fig.~\ref{fig:#1}}
\newcommand{\paper}{document}
%\newcommand{\sec}[1]{\S\ref{sec:#1}}

\title{Weak Lensing and Hierarchical Learning}
\author{Jacob T. VanderPlas}

\begin{document}
\maketitle

\begin{abstract}
  This document contains a brief description of weak lensing from the point
  of view of hierarchical graphical models.
\end{abstract}

\begin{figure}[htb]
  \centering
  \includegraphics[width=\textwidth]{lensing_simple.pdf}
  \caption{A simple PGM for the lensing problem}
  \label{fig:PGM_simple}
\end{figure}

%\keywords{}

\section{Introduction}
\label{sec:intro}

\fig{PGM_simple} shows a simple directed probabilistic graphical model for
the weak lensing problem.  The symbols denote the following quantities:
\begin{itemize}
  \item $\Omega$ is the parametrization of the cosmological model.  In the
    simplest case, this might be given by
    $\Omega = \{\Omega_M, \Omega_\Lambda, \sigma_8\}$.  Often, we will
    use a flatness prior such that $\Omega_M + \Omega_\Lambda = 1$.  In
    more sophisticated models, we may add in $w$, the equation of state
    of dark energy, or perhaps even $w_a$, which is related to the
    time-derivative of $w$.  These more complicated models will require
    more sophisticated shear observations (i.e. taking redshift into account).

  \item $M$ is the 3D continuous matter distribution throughout the universe.
    In the simplest case, $M$ can be thought of as a single realization of
    a three dimensional Gaussian random field with a covariance specified
    in terms of the cosmological parameters $\Omega$.

  \item $\gamma_{\rm true}^{(j)}$ is the true shear signal at the 3D position
    of the $j^{\rm th}$ galaxy.  Given perfect knowledge of $M$, it can
    be exactly specified from the field equations of General Relativity.

  \item $\Sigma$ is the distribution of intrinsic shape of the sources.  To
    first order, we can assume this noise is gaussian distributed and
    uncorrelated with the signal, with an RMS of $\sim 0.3$.

  \item $\epsilon_{\rm int}^{(j)}$ is the intrinsic shape of galaxy $j$, drawn
    from the distribution parametrized by $\Sigma$.

  \item $A$ is the generating model of the point-spread function for the
    images.  It includes stochastic contributions related to atmospheric
    turbulance, as well as non-stochastic contributions related to the
    telescope optics.  Note that even these non-stochastic contributions
    may change from exposure to exposure, because the sag of the telescope
    due to gravity depends on which way it is pointing.

  \item $PSF^{(i)}$ is the point-spread function at the location of the
    $i^{th}$ star.  These are stars in our galaxy which can be individually
    resolved: because stars are effectively point sources to high accuracy,
    the shape of the response on the CCD is a direct estimate of the PSF
    at that location.

  \item $PSF^{(j)}$ is the point-spread function at the location of the
    $j^{(th)}$ galaxy.  Galaxies are collections of billions of stars, and
    are too far away for us to resolve the stars individually.  This PSF
    is not observed, but affects the observation of the galaxy image.

  \item $\varepsilon^{(j)}_{\rm obs}$ is the observed shape of the galaxy.
    It can be read directly off the response of the CCD pixels, and
    in the weak lensing regime is linearly related to the intrinsic
    shape, the true shear, and the PSF.
\end{itemize}

\section{Detailed Relationships}
\label{sec:details}




%\bibliography{filter_info}

%\begin{appendix}
%\end{appendix}

\end{document}
