\documentclass[12pt]{article}
%\documentclass[preprint]{aastex}
%\documentclass[twocolumn]{emulateapj} 

\usepackage{graphicx}
\graphicspath{{../code/figures/}}

\newcommand{\eqn}[1]{eq.~(\ref{eq:#1})}
\newcommand{\eqns}[2]{eqs.~(\ref{eq:#1}) and (\ref{eq:#2})}
\newcommand{\fig}[1]{Fig.~\ref{fig:#1}}
\newcommand{\paper}{document}
%\newcommand{\sec}[1]{\S\ref{sec:#1}}

\title{Weak Lensing and Hierarchical Learning}
\author{Jacob T. VanderPlas}

\begin{document}
\maketitle

\begin{abstract}
  This document contains a brief description of weak lensing from the point
  of view of hierarchical graphical models.
\end{abstract}

\begin{figure}[htb]
  \centering
  \includegraphics[width=\textwidth]{lensing_simple.pdf}
  \caption{A simple PGM for the lensing problem}
  \label{fig:PGM_simple}
\end{figure}

%\keywords{}

\section{Introduction}
\label{sec:intro}

\fig{PGM_simple} shows a simple directed probabilistic graphical model for
the weak lensing problem.  The symbols denote the following quantities:
\begin{itemize}
  \item $\Omega$ is the parametrization of the cosmological model.  In the
    simplest case, this might be given by
    $\Omega = \{\Omega_M, \Omega_\Lambda, \sigma_8\}$.  Often, we will
    use a flatness prior such that $\Omega_M + \Omega_\Lambda = 1$.  In
    more sophisticated models, we may add in $w$, the equation of state
    of dark energy, or perhaps even $w_a$, which is related to the
    time-derivative of $w$.  These more complicated models will require
    more sophisticated shear observations (i.e. taking redshift into account).

  \item $M$ is the 3D continuous matter distribution throughout the universe.
    In the simplest case, $M$ can be thought of as a single realization of
    a three dimensional Gaussian random field with a covariance specified
    in terms of the cosmological parameters $\Omega$.

  \item $\gamma_{\rm true}^{(j)}$ is the true shear signal at the 3D position
    of the $j^{\rm th}$ galaxy.  Given perfect knowledge of $M$, it can
    be exactly specified from the field equations of General Relativity.

  \item $\Sigma$ is the distribution of intrinsic shape of the sources.  To
    first order, we can assume this noise is gaussian distributed and
    uncorrelated with the signal, with an RMS of $\sim 0.3$.

  \item $\epsilon_{\rm int}^{(j)}$ is the intrinsic shape of galaxy $j$, drawn
    from the distribution parametrized by $\Sigma$.

  \item $A$ is the generating model of the point-spread function for the
    images.  It includes stochastic contributions related to atmospheric
    turbulance, as well as non-stochastic contributions related to the
    telescope optics.  Note that even these non-stochastic contributions
    may change from exposure to exposure, because the sag of the telescope
    due to gravity depends on which way it is pointing.

  \item $PSF^{(i)}$ is the point-spread function at the location of the
    $i^{th}$ star.  These are stars in our galaxy which can be individually
    resolved: because stars are effectively point sources to high accuracy,
    the shape of the response on the CCD is a direct estimate of the PSF
    at that location.

  \item $PSF^{(j)}$ is the point-spread function at the location of the
    $j^{(th)}$ galaxy.  Galaxies are collections of billions of stars, and
    are too far away for us to resolve the stars individually.  This PSF
    is not observed, but affects the observation of the galaxy image.

  \item $\varepsilon^{(j)}_{\rm obs}$ is the observed shape of the galaxy.
    It can be read directly off the response of the CCD pixels, and
    in the weak lensing regime is linearly related to the intrinsic
    shape, the true shear, and the PSF.
\end{itemize}

\section{Procedural Solution}
The weak lensing problem is, roughly:

\begin{itemize}
\item{\bf Given the observed noisy galaxy shapes $\varepsilon^{(j)}_{\rm obs}$
  and the $PSF^{(i)}$ of field stars, compute the likelihood of the
  cosmological model specified by the parameters $\Omega$.}
\end{itemize}

Here we'll explain the steps of the procedural solution to the problem
which is typically used in the field

\subsection{Solve for the PSF}
The PSF has the form
\begin{equation}
  PSF(\Delta\vec{\theta}; \vec{\theta})
\end{equation}
where, for now, we'll assume it does not change with time.
That is, given a position on the sky $\vec{\theta}$ at time $t$, $P$
specifies where on the plane the light from a point source will be
spread to.  The form of the PSF must be parametrized somehow.  This can
be either a simple 2D gaussian fit, or perhaps a wavelet decomposition.
If we choose basis functions $p_k(\Delta\vec{\theta})$, then we may write
\begin{equation}
  PSF(\Delta\vec{\theta}; \vec{\theta}, t)
  = \sum_k A_k(\vec\theta) p_k(\Delta\vec{\theta}).
\end{equation}
It is helpful to choose the basis functions such that the $A_k$ are
uncorrelated: this is why a PCA-based is often used instead of a standard
wavelet basis.

Now this becomes a classic regression problem.  Given the parameters
$A_{ik} \equiv A_k(\vec\theta_i)$, we want estimates of the parameters
$A_{jk} \equiv A_k(\vec\theta_j)$.  This is often solved using a typical
maximum likelihood approach.

\subsection{Measure Galaxy Shapes}
For each galaxy $j$, our raw data is the pixel-level image of the galaxy.
Given this image along with the estimate of the PSF, we'd like to infer
the shear $\gamma_j$.

The distortion of the image is most conveniently expressed via the complex
shear
\begin{equation}
  \gamma \equiv \gamma_1 + i\gamma_2.
\end{equation}
This parametrization comes from the distortion matrix of the image,
\begin{equation}
  (1 + A) = \left(
  \begin{array}{cc}
     1 - \kappa - \gamma_1 & -\gamma_2\\
     -\gamma_2 & 1 - \kappa + \gamma_1
  \end{array}
  \right)
\end{equation}
where $\kappa$ is known as the {\it convergence} and $\gamma_{1, 2}$ are
the components of the shear.  To first order, this distortion is directly
related to the ellipticity $e$ of the measured image.
\begin{equation}
  e^{(j)}_{\rm obs} = PSF^{(j)} \ast (e^{(j)}_{\rm int} + \gamma)
\end{equation}
where $e^{(j)}_{\rm int}$ is the intrinsic ellipticity of the $j^{\rm th}$
galaxy.  Thus, assuming the intrinsic ellipticity is white noise, the
determination of $\gamma$ comes down to measuring the ellipticity of the
galaxy image, deconvolved with the PSF.

\subsection{Compute Cosmological Likelihood}
The next step involves taking these noisy shear estimates $\hat{\gamma}_j$,
and constraining the cosmological parameters $\Omega$.  The cosmological
parameters consist of things like the total matter density $\Omega_M$, the
dark energy density $\Omega_\Lambda$, the power spectrum normalization
$\sigma_8$, the dark energy equation of state $w$, the baryonic fraction
$\Omega_b$, and other parameters.  The details are interesting, but for
our purposes we can move forward using the mapping between cosmological
parameters and the matter power spectrum.

\subsubsection{The Details}
First we'll take a bit of a step back.  The distribution of matter in the
universe (what our model in \fig{PGM_simple} calls $M$) is given by the
density $\rho(\vec{x})$, usually measured in grams per cubic centimeter.
For cosmological purposes, we care about this density down to the scales
of galaxy groups or clusters, greater than, say, a million light years across.

For convenience we define the density contrast $\delta$, which is the
normalized deviation about the mean density:
\begin{equation}
  \delta(\vec{x}) = \frac{\rho(\vec{x}) - \bar{\rho}}{\bar{\rho}}.
\end{equation}
The correlation function is given by
\begin{equation}
  \xi_\delta(\vec{r}) = \langle \delta(\vec{x})\delta(\vec{x} + \vec{r})\rangle
\end{equation}
where angled brackets denote the average across all space.  The Fourier
transform of the correlation function is the power spectrum,
$P_\delta(\vec{k})$.  Because the distribution of matter is homogeneous and
isotropic, only the magnitude of the wave number $\vec{k}$ really matters.

But how is $\delta$ related to shear?  Well, it can be shown that the shear
power spectrum $P_\gamma(\ell)$ (here $\ell$ is the magnitude of a 2D angular
wave number, where $\vec{k}$ is a 3D spatial wave number) is related to
$P_\delta(k)$ by a simple integral:
\begin{equation}
  P_\gamma(\ell) = \int_0^{\inf} {\rm d}r W(r;\Omega) P_\delta(k = \ell/r)
\end{equation}
where $W(r)$ is a weight function which depends on the cosmology $\Omega$.

The interesting thing to note is that given a cosmological model $\Omega$,
it is possible to compute the functional form of $P_\delta(\vec{k})$, and
therefore possible to compute the power spectrum $P_\gamma(\ell)$.  In
practice, this involves some rather complicated empirical fits to simulations,
the details of which are not important right now.

\subsubsection{The Likelihood}
With all this in mind, then, given specified model parameters $\Omega$, we
can compute the expected power spectrum $P_\gamma(\ell)$ and compare it to
the observed power spectrum $\hat{P}_\gamma(\ell)$, which is generally computed
via its Fourier transform, the shear correlation function.
There are some more details involving the relationship between the multiple
possible correlation functions, E/B decompositions of the spin-2 shear
tensor, but again we can ignore those for now.

Thus the likelihood computation essentially short-circuits specification of
the mass distribution $M$.  The advantage of this is that the cosmological
observables can be directly related to the observed galaxy shapes through
the shear correlation function/power spectrum.  The disadvantage is that
the $\gamma_j$ are conditionally independent only if $M$ is specified.
When we perform the computation without specifying $M$, we must consider
all shear values at once when computing the likelihood of $\Omega$.

\section{Infinite Computational Power...}
What if we could sample $M$?  This is very difficult, of course,
because $M$ represents a continuous field $\rho(\vec{x})$ sampled
throughout space.

But suppose we had infinite computational power.  Then instead of working
backwards from $\varepsilon_{\rm obs}$ to $\Omega$, we could work forward
in a generative sense from $\Omega$ to $\varepsilon_{\rm obs}$.  Given
cosmological parameters $M$, we can compute the power spectrum of the
density $P_\delta(k)$.  Given $P_\delta(k)$ we could ``draw''
$\rho(\vec{x})$, a continuous scalar field throughout all of three dimensional
space.  Given this $\rho(\vec{x})$, each $\gamma_j$ is {\it completely}
specified, independent of the values of the other $\gamma_j$s.  We can
then compute the likelihood based on these estimates in an entirely
data-parallel manner, and marginalize over $M$ to arrive at our posterior
on $\Omega$.

The advantages here are that the computation is completely data parallel:
given a large cluster, it would be possible to evaluate each source
plane independently, computing the full marginalization over the PSF
model and the shape noise.  There would be no need to compute the
spatial correlation function or power spectrum of the observed shear.

The disadvantage, of course, is that it's entirely computationally infeasible.
Details, details...

%\bibliography{filter_info}

%\begin{appendix}
%\end{appendix}

\end{document}
